%!TEX root = main.tex

\begin{abstract}
Using passwords for user authentication is still the most common method for many
internet services and attacks on the password databases pose a severe threat. To
reduce this risk, servers store password hashes, which were generated using
special password-hashing functions, to slow down guessing attacks. The most
frequently used functions of this type are PBKDF2, bcrypt and scrypt.\\
%
In this paper, we present a novel, flexible, high-speed implementation of a
bcrypt password search system on a low-power Xilinx Zynq 7020 FPGA. The design
consists of 40 parallel bcrypt cores running at 100 MHz. Our implementation
outperforms all currently available implementations and improves password
attacks on the same platform by at least 42\%, computing 6,511 passwords per
second for a cost parameter of 5.
%
%We compare our results to other platforms, \ie CPUs, GPUs and a Xilinx Virtex-7
%FPGA, and evaluate the total cost for multiple attacking scenarios using the
%different platforms and implementations.
\end{abstract}

\section{Introduction}
\label{sec:intro}

In the modern world, we constantly use online services in our daily life. As a
consequence, we provide information to the corresponding service providers,
\eg financial services, email providers or social networks. To prevent abuse
like identity theft, we encounter access-control mechanisms at every step we
make. While it is one of the older mechanisms, password authentication is still
one of the most frequently used authentication methods on the internet even
with the emerging advanced login-procedures, \eg single sign-on or two-factor
authentications.
% passwords remain a necessary part.

To authenticate users for online services, these passwords are stored on
corresponding servers. As a consequence, an attack on these databases, followed
by a leak of the information, pose a very high threat to the users and may form
a single point of failure, if the passwords are stored in plain text. Recent
examples of such leaks are the eBay\footnote{\cf
\url{http://www.ebayinc.com/in_the_news/story/ebay-inc-ask-ebay-users-change-
passwords}} or Adobe\footnote{\cf \url{https://adobe.cynic.al/}} password leaks,
where several million passwords were stolen. To prevent these attacks or at
least raise the barrier of abuse, passwords must be protected on the server.
Instead of storing the password as plaint text, a cryptographic hash of the
password is kept. In this case, a successful attacker has to recover the
passwords from the hash value, which should in theory be infeasible due to the
properties of the hash function. To prevent time-memory trade-off techniques
like rainbow tables, the password is combined with a randomly chosen salt and
the tuple
\begin{equation*}
	(s, h) = (salt, \text{hash}(salt, password))
\end{equation*}
is stored. Improvements to exhaustive password searches with the aim to
determine weak passwords exists. As passwords are often generated from a
specific character set, \eg using digits, upper- and lower-case characters, and
may be length-restricted, \eg allowing six to eight characters, the search space
can be reduced considerably. This enables password recovery by brute-force or
dictionary attacks. Recently, more advanced methods,
\eg probabilistic context-free grammars~\cite{PWDCrackGrammars} or Markov
models~\cite{FastDictAttacks,OMEN} were analyzed to improve the password guesses
and the success rate and thus reduce the number of necessary guesses.

Apart from the generation of suitable password candidates, the implementation has a
high impact on the success. On general-purpose CPUs, generic tools like
\emph{John the Ripper} (JtR)\footnote{cf. \url{http://www.openwall.com/john}} or
target-specific tools like \emph{TrueCrack}\footnote{cf.
\url{http://code.google.com/p/truecrack}}, addressing a specific algorithm, in
this case TrueCrypt volumes, use algorithmic optimizations to gain a speedup
when testing multiple passwords. To further improve efficiency, not only the CPU
may be used: modern GPUs feature a large amount of parallel processing cores at
high clock-frequencies in combination with large memory. As a prominent example,
\emph{HashCat}\footnote{cf. \url{http://hashcat.net/oclhashcat}} utilizes this
platform for high-performance hash computations.

The major problem remains that hash functions are very fast to evaluate and thus
enable fast attacks. \emph{Password-hashing functions} address this issue. These
functions map a password to key material for further usage and explicitly slow
down the computation time by making heavy use of the available resources: the
computation should be fast enough to validate an honest user, but render
password guessing infeasible. One key idea to prevent future improvements in
architectures from breaking the efficiency of these function are flexible cost
parameters. These adjust the function in terms of time and/or memory complexity.

The current standardized password-based key-derivation function is PBKDF2 which
is part of the Public-Key Cryptography Standards (PKCS)~\cite{RFC2898}.
Non-standardized alternatives are bcrypt~\cite{BCRYPT_paper} and
scrypt~\cite{percival-09-scrypt}.
%Both are frequently used and feature different security approaches, \eg
%requiring very large memory or frequently accessing small memory randomly.
While the three functions are considered secure, each has its own advantages
and disadvantages. This lead to the currently running password hashing
competition (PHC)\footnote{\cf \url{https://password-hashing.net/}}, which aims
at providing well-analyzed alternatives. Another purpose is discussing new ideas
and different security models with respect to the impact of special-purpose
hardware like modern GPUs, Application-Specific Integrated Circuits
(ASICs) or Field-Programmable Gate Arrays (\mbox{FPGAs}).

Usually, the overall cost of large-scale attacks on cryptographic functions --
and thus the feasibility of the attack -- is dominated by the power costs. For
this reason, specialized hardware achieves excellent results due to its low
power consumption, especially when compared to general-purpose architectures.
This makes special-purpose hardware very attractive for cryptanalysis in
general~\cite{CryptanalysisCOPA, EnhancingCOPA, RealWorldA51, FPGAFactoring},
as well as in the context of password-hashing functions~\cite{PBKDEvalutation}.

For the remainder of this paper, we focus on bcrypt as the target function in
the scope of efficient password-guessing attacks and start with an overview on
the currently available implementations. In Section~\ref{sec:bcrypt}, we describe the bcrypt
algorithm and the influence its tunable cost parameter.

With the goal of benchmarking energy-efficient password cracking,
\cite{PWDCON_Bcrypt} provided several implementations of bcrypt on low-power
devices, including an FPGA implementation in December 2013. The authors used the
zedboard (cf. \ref{sec:special-hardware}), which combines an ARM processor and
an FPGA, and split the workload on both platforms. The FPGA computes the
time-consuming cost-loop of the algorithm while the ARM manages the setup and
post-processing. They reported up to 780 passwords per second (pps) for a cost
parameter of 5 and identified the highly unbalanced resource usage as a drawback
of the design.
%
In August 2014, \cite{WOOT/Malvoni14} presented
%\footnote{The article was
%accepted at WOOT'14 and will be presented on August 19, 2014. We were notified
%about the results and provided with information about the content of the final
%version. This paragraph and the numbers may be updated once the paper and
%presentation are public.}
a new design, improving the performance to 4571 pps
for the same device and parameter, using the ARM only for JtR to generate
candidates and to transfer initialization values to the FPGA. When they further
optimized performance, the zedboard became unstable (heat and voltage problems).
Due to these issues they also report a higher theoretical performance of 8122
pps (derived from cost 12) and 7044 pps (simulated using the larger Zynq 7045
FPGA).

\emph{Contribution:} In this paper, we provide a practical and efficient
implementation of bcrypt on a low-power FPGA-platform. Compared to the previous
implementations on the same device, we achieve a performance gain of 8.35 and
1.42, respectively.
%, and outperform the theoretical upper bounds by more than 60\%.
In addition, we implemented a simple on-chip password generation to
utilize free area in the fabric, which splits a pre-defined password space and
generates all possible brute-force candidates. This creates a self-contained,
fully functional system (which may still use other sources for password
candidate checking), which we compare to other currently available
attack-platforms.

\emph{Outline:} The rest of the paper is structured as follows: We first
introduce the necessary background information in Section~\ref{sec:background},
before we describe our implementation details in the subsequent section and
discuss our results, followed by an evaluation of the costs of different attack
scenarios, in Section~\ref{sec:results}. Finally, our conclusion and
perspectives for future work form the last section.
